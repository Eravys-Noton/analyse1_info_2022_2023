\chapter{Nombres réels}

\section{Les ensembles de nombres}

Le premier ensemble de nombres est l'ensemble 
\[ \N = \{0,1,2,\dots\} \] 
de tous
les \emph{entiers naturels}. Une variante est $\N^* = \{1,2,3,\dots \}$, l'ensemble des entiers naturels non nuls. L'ensemble $\N^*$ correspondant à l'ensemble $\N$ auquel on a enlevé $0$.

Le second ensemble de nombres est l'ensemble
\[ \Z = \{ \dots, -3,-2,-1,0,1,2,3, \dots \} \]
de tous les \emph{entiers relatifs}. On définit de même $\Z^*$ comme l'ensemble $\Z$ auquel on a enlevé (privé de) $0$.

L'ensemble $\Z$ est plus gros que $\N$. Ceci s'écrit en formule comme
\[ \N \subset \Z .\]
Le symbole $\subset$ est le symbole d'\emph{inclusion}, qui se lit «est inclus dans». Si $A$ et $B$ sont des ensembles, on écrit $A \subset B$ lorsque tout élément de $A$ est un élément de $B$. On a $\Z \not \subset \N$, c'est-à-dire que $\Z$ n'est pas inclus dans $\N$, puisque par exemple le nombre $-1$ est dans $\Z$ mais pas dans $\N$.

Il faut faire attention à ne pas confondre le symbole $\subset$ avec le symbole $\in$ qui lui ressemble. Le symbole $\in$ se lit «appartient à». Par exemple, la formule « $n \in \N$ » se lit « $n$ appartient à $\N$ », c'est-à-dire que $n$ est l'un des entiers qui sont dans la liste $\{0,1,2,\dots,\}$.

On a ainsi\footnote{A l'inverse, une formule comme \textcolor{red}{$5 \subset \N$} n'est ni vraie ni fausse, elle n'a \emph{pas de sens} - on l'écrit en rouge pour insister. Un correcteur lisant cette formule va entrer dans le mode "syntax error !", ce qui est en général mauvais signe pour l'étudiant dont c'est la copie}
\[ \N \subset \Z , \ \ \ \Z \not\subset \N \]
\[ 3 \in \N , \ \ \ 3 \in \Z , \ \ \ -4 \in \Z, \ \ \, -4 \not\in \N .\]

On continue notre visite du zoo des ensembles de nombres avec l'ensemble $\Q$ de tous les \emph{nombres rationnels}, c'est-à-dire de toutes les \emph{fractions}. Ainsi
\[ \Q = \left\{ \textnormal{quotients de la forme } \frac{a}{b} \textnormal{ avec } a \in \Z,\ b \in \Z^* \right\}.\]
(On a pris soin de préciser $b \in \Z^*$ et non $b \in \Z$ afin de ne pas diviser par $0$).

\begin{remarque}
	On aurait pu aussi définir $\Q$ en demandant $a \in \N$, $b \in \Z^*$. Vois-tu pourquoi ?
\end{remarque}

Cette définition de $\Q$ pose une difficulté : en effet, un même nombre rationnel peut être représenté par plusieurs fractions. Par exemple, on a $\frac 12 = \frac 24$. On peut décrire précisément quand cela arrive. En effet, si $a \in \Z$, $b \in \Z^*$, $a' \in \Z$, $b' \in \Z$, on a
\[ \frac{a}{b} = \frac{a'}{b'} \textnormal{ si et seulement si } ab'=a'b .\]

Les mathématiciens utilisent souvent l'expression «si et seulement si», que tu peux écrire «ssi». Il faut bien comprendre son sens. Quand on écrit
\[ P \textnormal{ si et seulement si }Q ,\]
cela veut dire deux choses à la fois. Tout d'abord,
\[ \textnormal{ si }P\textnormal{ est vraie, alors }Q\textnormal{ est vraie},\]
mais également
\[ \textnormal{ si }Q\textnormal{ est vraie, alors }P\textnormal{ est vraie}.\]

Il y a un ensemble de nombres qu'il ne faut pas confondre avec $\Q$, c'est l'ensemble $\D$ de tous les nombres décimaux. Un nombre \emph{décimal} est un nombre de la forme $\frac{a}{10^n}$ avec $a \in \Z$ et $n \in \N$. Un nombre décimal s'écrit avec une suite finie de chiffres à droite de la virgule.

Par exemple le nombre $46,253$ est un nombre décimal car il peut s'écrire comme $\frac{46253}{10^3}$. Il est clair que tout nombre décimal est rationnel, c'est à dire que $\D \subset \Q$. Cette inclusion est \emph{stricte}, c'est-à-dire que $\D \not= \Q$. En effet, il existe des rationnels non décimaux, comme la fraction
%jai changé le notsubset en not= psk ct pas ecrit comme ca dans le cours + ca pas de sens ici tu dis inclus apres pas inclus.Racem Grab
\[ \frac{1}{3} = 0,\underbrace{333333333333333333333\dots}_{\textnormal{suite infinie de }3} \]
Pour les mathématiciens, l'ensemble $\Q$ est plus important que l'ensemble $\D$; en effet ce dernier fait jouer un rôle privélégié au nombre $10$ sans justification autre que le fait que \emph{Homo Sapiens} possède $10$ doigts.

On arrive maintenant à l'ensemble qui donne le nom à ce chapitre, l'ensemble $\R$ des nombres réels. Un nombre réel est un nombre comme par exemple
\[ -75,2828746\dots \]
qui s'écrit en mettant bout à bout
\begin{itemize}
	\item un signe $+$ ou $-$, que l'on sous-entend généralement lorsqu'il s'agit de $+$,
	\item un nombre entier,
	\item une virgule,
	\item une suite finie ou infinie de chiffres après la virgule.
\end{itemize}

Comme pour les rationnels, une difficulté vient du fait qu'un même nombre réel peut avoir plusieurs telles écritures. On a ainsi
\[ +0 = -0 = 3.5 = 3.50000 \] % jai modifié ca, ca me semble plus parlant comme ca si tu trouves pas tu peux remettre ton exemple ou alors dit moi et jle changerais
\[ 1 = 0,\underbrace{99999999999999999999999999999999999\dots}_{\textnormal{suite infinie de }9} \]
On n'insistera pas sur ces difficultés, qui seraient très problématiques si on voulait donner des \emph{preuves} des propositions de la section suivante. Tu peux aller lire l'article 
\url{https://fr.wikipedia.org/wiki/Construction_des_nombres_r%C3%A9els}
pour découvrir des façons plus efficaces que celle décrite ci-dessus d'introduire l'ensemble $\R$.

\section{Opérations et relation d'ordre dans les réels}

\`A l'école primaire, on apprend à ajouter, multiplier et comparer les entiers naturels. Ceci s'étend aux réels.

\begin{graybox}
	\begin{proposition}[Addition et multiplication sur $\R$]
		On peut définir sur $\R$ une addition (notée $+$) et une multiplication (notée $\times$ ou $\cdot$) qui prolonge l'addition et la multiplication de $\N$ et vérifie les règles suivantes.
		\begin{enumerate}
			\item (Commutativité) pour tous $a$, $b$ dans $\R$, on a
			\[ a+b=b+a \textnormal{ et } a \cdot b =b \cdot a.\]
			\item (Associativité) pour tous $a$, $b$, $c$ dans $\R$, on a
			\[ a+(b+c) = (a+b)+c \textnormal{ et } a \cdot (b \cdot c) = (a \cdot b) \cdot c .\]
			\item (Distributivité) pour tous $a$, $b$, $c$, on a
			\[ (a+b) \cdot c = a \cdot c + b \cdot c .\]
			\item (Éléments neutres ou absorbants) pour tout $a$ dans $\R$, on a
			\[ a+0 = a , \ \ a \cdot 1 =a , \ \ a \cdot 0 =0 .\]
		\end{enumerate}
	\end{proposition}
\end{graybox}


Démontrer la proposition serait fastidieux : il faudrait décrire un algorithme qui explique comment ajouter et multiplier deux nombres réels, puis vérifier toutes ces propriétés.


\begin{graybox}
	\begin{proposition}[Relation d'ordre sur $\R$]
		On peut définir sur $\R$ une relation d'ordre, notée $\leq$, qui prolonge l'ordre de $\N$ et vérifie les règles suivantes.
		\begin{enumerate}
			\item (Réflexivité) pour tout $a$ dans $\R$, on a
			\[ a \leq a .\]
			\item (Antisymétrie) pour tous $a,b$ dans $\R$,
			\[ \textnormal{ si } a \leq b \textnormal{ et } b \leq a, \textnormal{ alors } a=b. \]
			\item (Transitivité) pour tous $a,b,c$ dans $\R$,
			\[ \textnormal{ si } a \leq b \textnormal{ et } b \leq c, \textnormal{ alors } a \leq c. \]
			\item (Ordre total) pour tous $a,b$ dans $\R$, on a
			\[ a \leq b \textnormal{ ou}\footnote{pour les mathématiciens, le «ou» n'est pas exclusif, c'est-à-dire que «$P$ ou $Q$» veut dire: soit $P$, soit $Q$, soit à la fois $P$ et $Q$. Quand on dit «ou» dans la vie courante, c'est souvent implicitement exclusif. Ainsi un mathématicien pourra répondre «oui\space!» à des questions telles que «Est-ce une fille ou un garçon ?», «Fromage ou dessert ?», «Vous montez ou vous descendez ?»} b \leq a .\]
			\item (Compatibilité avec l'addition) pour tous $a,b,c$ dans $\R$,
			\[ \textnormal{ si } a \leq b \textnormal{ alors } a+c \leq b+c. \]
			\item (Compatibilité avec la multiplication par un réel {\bfseries positif}) pour tous $a,b,c$ dans $\R$,
			\begin{align*}
				\textnormal{ si } a \leq b \textnormal{ et } \framebox{$c \geq 0$}, \textnormal{ alors } a\cdot c \leq b\cdot c.
			\end{align*} 
		\end{enumerate}
	\end{proposition}
\end{graybox}

% Début : Raphaël Heng 16/09/22
\section{Valeur absolue}

\begin{graybox}
	\begin{definition}[Valeur absolue]
		Soit $x \in \R$, la valeur absolue, notée $|x|$ est définie ainsi :
		\begin{align*}
			\mid x \mid = \begin{cases}
				x &\textnormal{ si } x \geq 0 \\
				-x &\textnormal{ si } x < 0
			\end{cases}
		\end{align*}
	\end{definition}
\end{graybox}

\begin{graybox}
	\begin{proposition}
		La valeur absolue vérifie les propriétés suivantes, pour $\forall$ a,b $\in \R$ :
		\begin{enumerate}
			\item $|a + b| \leq |a| + |b| \textnormal{ : \textbf{(Inégalité triangulaire)}}$
			\item $|a \cdot b| = |a| \cdot |b|$
			\item $|a - b| \geq |a| - |b| \textnormal{ :\textbf{(Inégalité triangulaire inverse)}}$
			\item $|a|= \sqrt{a^2}$
		\end{enumerate}
	\end{proposition}
\end{graybox}


% Fin : Raphaël Heng 16/09/22

%Debut : Racem Grab 16/09/2022

\section{Intervalles de $\R$}
Un intervalle est une partie de $\R$ (c'est-à-dire dans $\R$) sans "trou"

\begin{graybox}
	\begin{definition}[Intervalle]
		Soit I $\subset \R$. On dit que I est un intervalle 
		$$ \textnormal{Si } \forall (x, y) \in I^2, \textnormal{ alors } z \in I \textnormal{ et } z \in \R,\ x \leq z \leq y $$
	\end{definition}
\end{graybox}

\begin{graybox}
	\begin{proposition}[Intervalles de $\R$]
		Les intervalles de $\R$ sont de type suivant :
		\begin{itemize}
			\item $\R$
			\item $[a,+\infty[$ \textnormal{ou} $]a, +\infty[$
			\item $[a,b]$ \textnormal{ou} $[a,b[$ \textnormal{ou} $]a,b[$ \textnormal{ou} $]a,b]$ \textnormal{(a, b)} $\in \R$  \textnormal{avec a < b} $\in \R$ 
			%pour le < je suis pas sur jarrive pas a savoir si cest un inclus ou un < si tu peux me corriger stp merci.Racem : C'est bien un <, merci pour ton aide, Raphaël. Pas de soucis merci a toi aussi !
			\item $]-\infty, a]$ \textnormal{ou} $]-\infty , a[$ \textnormal{avec a $\in \R$}
			\item $\emptyset$, \textnormal{l'ensemble vide}
			\item $\{ a \}$ \textnormal{où a} $\in \R$ \textbf{(un singleton)}
		\end{itemize}
		\medskip 
		On a
		\newline
		$[a,b[$ = $\{x \in \R ~ tel ~ que ~ a \leq x < b \}$
		\newline
		$]-\infty, a]$=$\{ x \in \R ~ x \leq a \}$
	\end{proposition}
\end{graybox}


\section{Borne inférieure, borne supérieure}

Soit A $\subset \R$ une partie de $\R$ et m $\in \R$, un élément de $\R$
\newline
On dit que m est un majorant de A si
\begin{align*}
	\forall x \in A,\ x \leq m
\end{align*}
On dit que m est un minorant de A si
\begin{align*}
	\forall x \in A,\ x \geq m
\end{align*}
\noindent On dit qu'une partie de A $\subset \R$ est \textbf{majorée} si elle admet un \textbf{majorant}
\newline 
On dit qu'une partie de A $\subset \R$ est \textbf{minorée} si elle admet un \textbf{minorant}
\newline 
On dit qu'une partie de A $\subset \R$ est \textbf{bornée} si elle est \textbf{majorée et minorée}
\newline
%Fin Racem Grab 16/09/2022

% Raphaël Heng 16/09/22

\begin{graybox}
	\begin{theoreme}[Théorème de la borne supérieure]
		Toute partie $A \subset \R$ \textbf{non-vide} et \textbf{majorée} admet un \textbf{plus petit majorant} appelé la \textbf{borne supérieure} de A, notée : $\textnormal{sup}(A)$
	\end{theoreme}
\end{graybox}

\begin{graybox}
	\begin{theoreme}[Théorème de la borne inférieure]
		Toute partie $A \subset \R$ \textbf{non-vide} et \textbf{minorée} admet un \textbf{plus grand minorant} appelé la \textbf{borne inférieure} de A, notée : $\textnormal{inf}(A)$
	\end{theoreme}
\end{graybox}

\begin{graybox}
	\begin{proposition}[]
		Par convention : \\
		\begin{itemize}
			\item Si A une partie de $\R$ non-vide mais pas majorée : $\textnormal{sup}(A) = +\infty$.
			\item Si A une partie de $\R$ non-vide mais pas minorée : $\textnormal{inf}(A) = -\infty$.
		\end{itemize}
	\end{proposition}
\end{graybox}

\begin{remarque}
	Il n'y a pas de théorème sur les bornes dans $\Q$.
\end{remarque}

\begin{graybox}
	\begin{proposition}[Caractérisation de la borne supérieure]
		Soit $A \subset \R$ non-vide et $m$ un majorant de $A$. 
		\begin{align*}
			m = \textnormal{sup}(A) \iff \forall \varepsilon > 0, ]m - \varepsilon, m] \cap A \neq \emptyset
		\end{align*}
		$\varepsilon$ : \textnormal{Lettre grecque , "Epsilon" utilisée pour désigner un réel très petit.} %modification de nombre en réel.Racem
		\newline
		$\cap$ : \textnormal{désigne qu'il a une intersection entre deux parties, (éléments en communs)}     
	\end{proposition}
\end{graybox}

\begin{exemple}
	Si x $\in R$
	\begin{itemize}
		\item \textnormal{A = [0,2[} \textnormal{x} $\in$ \textnormal{A ssi x} $\geq$ \textnormal{0 et x < 2}
		\item \textnormal{B = ]1,3[} \textnormal{x} $\in B$ \textnormal{ssi x > 1 et x <3}
		\item \textnormal{A} $\cap$ \textnormal{B} \textnormal{ssi (x} $\geq$ \textnormal{0) et x < 2 ici et 1 < x (et x <3) donc A}$\cap$ \textnormal{B = ]1,2[} 
	\end{itemize}
\end{exemple}

