\chapter{Suites réelles}
\section{Définitions}

\begin{graybox}
    \begin{definition}[Suite réelle]
\par On appelle \textbf{suite réelle} une fonction de $\N \to \R$. On note $(u_n)_{n \in \N}$ la fonction $x \mapsto u_n$
\end{definition}
\end{graybox}

\begin{exemple}
\par On définit la suite $(u_n)_{n \in \N}$ par la formule 
\begin{align*}
    \forall n \in \N, u_n = 3n + 7
\end{align*}

\begin{itemize}
    \item $u_0 = 7$
    \item $u_1 = 10$
    \item $u_2 = 13$
\end{itemize}
\end{exemple}

\begin{remarque}
\par On met des parenthèses pour parler de la suite dans son intégralité. On ne met pas de parenthèses pour parler d'un seul terme de la suite.
\end{remarque}

\begin{remarque}
    \par On peut définir une suite par récurrence.
\end{remarque}

\begin{exemple}
    \begin{align*}
        \forall n \geqslant 1 :
        \begin{cases}
        u_0 = 2 \\
        u_n = \frac{1}{n} + \frac{u_{n - 1}}{3}
        \end{cases}
    \end{align*}
\end{exemple}

\begin{remarque}
    Le vocabulaire des fonctions s'applique aussi aux suites.
\end{remarque}

\begin{graybox}
    \begin{definition}[Monotonie d'une suite]
    \par Soit $(u_n)_{n \geqslant 0}$ une suite. 
    \begin{itemize}
        \item On dit que $(u_n)_{n  \geqslant 0}$ est \textbf{croissante} si $\forall n \in \mathbb{N}, u_n \leqslant u_{n + 1}$
        \item On dit que $(u_n)_{n \geqslant 0}$ est \textbf{décroissante} si $\forall n \in \mathbb{N}, u_n \geqslant u_{n+1}$
        \item $(u_n)_{n \geqslant 0}$ est \textbf{majorée} si $\exists M \in \mathbb{R}, \forall n \in \mathbb{N}, u_n \leqslant M$
        \item $(u_n)_{n \geqslant 0}$ est \textbf{minorée} si $\exists m \in \mathbb{R}, \forall n \in \mathbb{N}, u_n \geqslant m$
        \item $(u_n)_{n \geqslant 0}$ est \textbf{bornée} si elle est majorée et minorée $\iff \exists M \in \mathbb{R}, \forall n \in \mathbb{N}, |u_n| \leqslant M$
    \end{itemize}
\end{definition}
\end{graybox}

\begin{remarque}
    \par Une suite $(u_n)_{n \geqslant 0}$ est croissante $\iff$ $\forall (m,n) \in \mathbb{N}^2, m \leqslant n \implies u_m \leqslant u_n$
    On dit aussi que $(u_n)_{n \geqslant 0}$ est monotone si elle est croissante ou décroissante.
\end{remarque}

\begin{remarque}
    \par L'ensemble d'indices est parfois $\mathbb{N}^*$ plutôt que $\mathbb{N}$, d'où la suite $(u_n)_{n \geqslant 1}$.
    On peut aussi avoir $(u_n)_{n \geqslant 2}$. Les résultats du cours seront énoncés par des suites $(u_n)_{n \geqslant 0}$ mais aussi valables pour des suites $(u_n)_{n \geqslant 1}$
\end{remarque}

\begin{remarque}
    \par Soit $P(x)$ une propriété qui dépend de $n \in \mathbb{N}$. On dit que $P(n)$ est vraie \textbf{à partir d'un certain rang}, si $\exists n_0 \in \mathbb{N}, \forall n \geqslant n_0$, $P(n)$ est vraie.
\end{remarque}

\begin{exemple}
    \par Soit $(u_n)_{n \geqslant 0}$ définie par $\forall n \in \mathbb{N}, u_n = 2^n - 10n$
    \begin{itemize}
        \item $u_0 = 1$
        \item $u_1 = -8$
        \item $u_2 = -16$
        \item $u_3 = -22$
    \end{itemize}
    \par La suite $(u_n)_{n \geqslant 0}$ n'est pas croissante. Mais elle est croissante à partir d'un certain rang.
    \begin{align*}
        \forall n \in \mathbb{N}, u_{n+1} - u_n &= 2^{n+1} - 10(n+1) - 2^n + 10n \\
        &= u_{n+1} - u_n = 2 \cdot 2^n - 2^n - 10 \\
        &= u_{n + 1} - u_n = 2^n - 10 \text{ qui est } \geqslant 0 \text{ pour } n \geqslant 4
    \end{align*}
\end{exemple}

\begin{graybox}
    \begin{proposition}[Propriétés des suites]
    \par Soient $(u_n)_{n \geqslant 0} \text{ et } (v_n)_{n \geqslant 0}$ deux suites. On peut former :
    \begin{itemize}
        \item La somme : $(u_n + v_n)_{n \geqslant 0}$
        \item Le produit : $(u_n v_n)_{n \geqslant 0}$
        \item Pour $\lambda \in \mathbb{R}, (\lambda u_n)_{n \geqslant 0}$
    \end{itemize}
\end{proposition}
\end{graybox}

\subsection{Suites classiques}

\begin{graybox}
    \begin{proposition}[Suite arithmétique]
    Suite arithmétique de progression $r \in \R$ donnée par $u_0 \in \mathbb{R}$ est la relation de récurrence $u_{n+1} = u_n + r, \forall n \in \mathbb{N}$. On a alors pour tout $n \in \mathbb{N}$
    \begin{align*}
        u_n = u_0 + nr
    \end{align*}
    On peut aussi calculer :
    \begin{align*}
        \sum_{k = 0}^{n} u_k &= \sum_{k=0}^{n} (u_0 + kr) \\
        &= (n+1) u_0 + r\sum_{k=0}^{n}k \\
        &= (n+1)u_0 + r \frac{n(n+1)}{2}
    \end{align*}
\end{proposition}
\end{graybox}

\begin{graybox}
    \begin{proposition}[Suite géométrique]
    Suite géométrique de raison $q \in \R^*$ donnée par $u_0 \in \mathbb{R}$ et la relation de récurrence $u_{n+1} = q u_n, \forall n \in \mathbb{N}$
	       On a alors $\forall n \in \mathbb{N}, u_n = u_0 q^n$
	       On peut aussi calculer :
        	\begin{align*}
        	\sum_{k=0}^{n}u_k &= \sum_{k=0}^{n} u_0 q^k \\
        					  &= u_0 \sum_{k=0}^{n}q^k	
        	\end{align*}        
            Sachant que : 
            \begin{align*}
                	\sum_{k=0}^{n} q^k = \frac{1 - q^{n+1}}{1 - q}, \ q\neq 1
            \end{align*}
            On a :
            	\begin{align*}
            	&\text{Si }q\neq 1 :  \sum_{k=0}^{n}u_k = u_0 \frac{1 - q^{n+1}}{1 - q} \\
            	&\text{Si } q = 1 : \sum_{k=0}^{n}u_k = (n+1)u_0
            	\end{align*}
\end{proposition}
\end{graybox}

\begin{graybox}
    \begin{proposition}[Suite arithmético-géométrique]
    Suites arithmético-géométrique de progression $r \in \mathbb{R}$ et de raison $q \in \mathbb{R^*}$ donnée par $u_0 \in \mathbb{R}$ et $\forall n \in \mathbb{N}, u_{n+1} = qu_n + r$ Comment calculer $u_n$ ?
        \begin{itemize}
            \item Si $q = 1$ c'est une suite arithmétique
            \item Si $q \neq 1$, on cherche le réel $a$ tel que $a = q a + r$ et on regarde la suite $(u_n - a)_{n \geqslant 0}$
            	 \begin{align*}
                	 a = qa + r &\iff a - qa = r \\
                	            &\iff a = \frac{r}{1 - q} \text{ (Possible 
                 car } q\neq 1)
              \end{align*} 
              
              $\forall n \in \N$ :
              	\begin{align*}
            	u_{n+1} &= qu_n + r \\
            	u_{n+1} - a &= qu_n + r - a \\
            	v_{n+1} &= q(v_n + a) + r - a \\
            	        &= qv_n + \underbrace{qa + r - a}_{=0}
            	\end{align*}
             Ainsi $(v_n)_{n \geq 0}$ est une suite géométrique de raison $q$, donc $\forall n \in \N$
             	\begin{align*}
            	v_n &= v_0 q^n \\
            	    &= (u_0 - a) q^n
            	\end{align*}
              et 
                  \begin{align*}
                    u_n &= v_n + a \\
                    u_n &= a + (u_0 - a)q^n
                    \end{align*}
        \end{itemize}   
\end{proposition}
\end{graybox}

\section{Convergence d'une suite}

\begin{graybox}
    \begin{definition}[]
    \par Soit $(u_n)_{n \geqslant 0}$ une suite et $\ell \in \mathbb{R}$. On dit que la suite $(u_n)_{n \geqslant 0}$ tend vers $l$ si :
    \begin{align*}
    \forall \varepsilon > 0,\ \exists N \in \mathbb{N},\ \forall n \geqslant N,\ |u_n - \ell| \leqslant \varepsilon
    \end{align*}
    \par On note alors :
    \begin{align*}
    u_n \xrightarrow[n \to +\infty]{}\ell \ \text{ ou }\lim_{n \to +\infty} u_n = \ell
    \end{align*}
\end{definition}
\end{graybox}



    \begin{remarque}
    $\varepsilon$ : epsilon s'utilise pour un réel > 0 petit
\end{remarque}



    \begin{remarque}
    La valeur de N dépend de $\varepsilon$ car le $\exists$ vient après le $\forall$
\end{remarque}



    \begin{remarque}
    $|u_n - \ell| \leqslant \varepsilon$ revient à dire que $-\varepsilon \leqslant u_n - \ell \leqslant \varepsilon$
    ou encore $\ell - \varepsilon \leqslant u_n \leqslant \ell + \varepsilon$
    ou bien $u_n \in [\ell - \varepsilon, \ell + \varepsilon]$
    Ainsi, dire que  $\displaystyle{\lim_{n \to +\infty} u_n = \ell}$ revient à dire que tout intervalle $[\ell-\varepsilon, \ell + \varepsilon]$ contient les termes de $(u_n)_{n \geqslant 0}$ à partir d'un certain rang.
\end{remarque}



    \begin{exemple}
    La suite $(u_n)_{n \geqslant 0}$ définie par $u_n = \frac{1}{n+1}, \forall n \in \mathbb{N}$ tend vers 0.
\end{exemple}



    \begin{proof}
    Soit $\varepsilon > 0$ 
	Si $\ell = 0$
	\begin{align*}
	|u_n - \ell| \leqslant \varepsilon &\iff \left| \frac{1}{n+1} \right| \leqslant \varepsilon \\
	&\iff \frac{1}{n+1} \leqslant \varepsilon \\
	&\iff \frac{1}{\varepsilon} \leqslant n + 1
	\end{align*}
    Posons $N = E(\frac{1}{\varepsilon})$ alors $N \leqslant \frac{1}{\varepsilon} < N + 1$
    et alors : 
    \begin{align*}
        \forall n \geqslant N, n + 1 \geqslant N + 1 > \frac{1}{\varepsilon}
    \end{align*}
    donc $n + 1 \geqslant \frac{1}{\varepsilon}$ et $|u_n - \ell| < \varepsilon$
    On a bien vérifié que $\displaystyle{\lim_{n \to +\infty} u_n = 0}$
    \end{proof}


    \begin{remarque}
    Pour démontrer une assertion de type $\forall \varepsilon > 0 \ldots$ on commence par "Soit $\varepsilon > 0$" 
\end{remarque}



    \begin{exemple}
    Soit $(u_n)_{n \geqslant 0}$ définie par :
    \begin{align*}
        u_n =
        \begin{cases}
            0 \ &\text{ si n pair } \\
            \frac{1}{n} \ &\text{ si n impair}
        \end{cases}
    \end{align*}
    La suite tend vers 0
    \begin{proof}
        Soit $\varepsilon > 0$
        Posons : $N = E \left(\frac{1}{\varepsilon} \right) + 1$, alors $\frac{1}{\varepsilon} \leqslant N$
        \\
        Soit $n \geqslant N$
        \begin{itemize}
            \item Si n est pair : $\left| u_n - 0 \right| = 0 < \varepsilon$
            \item Si n est impair : $\left|u_n - 0 \right| = \frac{1}{n} < \frac{1}{N} \leqslant \varepsilon$
        \end{itemize}
        Et donc $\displaystyle{\lim_{n \to +\infty} u_n = 0}$
    \end{proof}
\end{exemple}

\begin{graybox}
    \begin{definition}[]
    On dit que $(u_n)_{n \geqslant 0}$ converge si $\exists \ell \in \mathbb{R} \backslash \{\pm\infty\}$ tel que $\displaystyle{\lim_{n \to +\infty} u_n = \ell}$
    \begin{align*}
        (u_n)_{n \geqslant 0} \text{ converge} \iff \exists \ell \in \mathbb{R}, \forall \varepsilon > 0, \exists N \in \mathbb{N}, \forall n \geqslant N, \left|u_n - \ell\right| \leqslant \varepsilon
    \end{align*}
    Sinon on dit que $(u_n)_{n \geqslant 0}$ diverge
    \begin{align*}
        (u_n)_{n \geqslant 0} \text{ diverge } \iff \forall \ell \in \mathbb{R}, \exists \varepsilon > 0, \forall N \in \mathbb{N}, \exists n \geqslant N, \left|u_n - \ell \right| > \varepsilon
    \end{align*}
    \begin{align*}
        (u_n)_{n \geqslant 0} \text{ ne converge pas vers  }\ell &\iff \exists \varepsilon > 0, \forall N \in \mathbb{N}, \exists n \geqslant N, |u_n - \ell| > \varepsilon \\
        &\iff \exists \varepsilon > 0, \{n \in \mathbb{N} \text{ tq } |u_n - \ell| > \varepsilon\} \\ &\text{ est un ensemble infini}
    \end{align*}
    \begin{align*}
        \lim_{n \to +\infty} = \ell \iff \forall \varepsilon > 0 \{n \in \mathbb{N}, |u_n - \ell| > \varepsilon\} \text{ est fini }
    \end{align*}
\end{definition}
\end{graybox}

\begin{exemple}
    Soit $(u_n)_{n \geqslant 0}$ définie par 
    \begin{align*}
        u_n = (-1)^n =
        \begin{cases}
        1 &\text{ si n pair} \\
        -1 &\text{ si n impair}
        \end{cases}
    \end{align*}
    Cette suite diverge. \\
    En effet, soit $\ell \in \mathbb{R}$
    \begin{itemize}
        \item Si $\ell < 0$
        \begin{align*}
            |u_n - \ell| = |1 - \ell| \geqslant 1
        \end{align*}
        quand n est pair et donc aucun nombre pair n vérifie
        \begin{align*}
            |u_n - \ell| \leqslant \frac{1}{2}
        \end{align*}
        et donc $(u_n)_{n \geqslant 0}$ ne converge pas vers $\ell$
        puisque l'ensemble  
        \begin{align*}
            \left\{n \in \mathbb{N}, |u_n - \ell| > \frac{1}{2}\right\}
        \end{align*}
        est infini.
        
        \item Si $\ell > 0$, de même $|u_n - \ell| \geqslant 1$ pour n impair et donc $(u_n)_{n \geqslant 0}$ ne converge pas vers $\ell$
    \end{itemize}
\end{exemple}

\begin{graybox}
    \begin{theoreme}[Unicité de la limite]
    La limite d'une suite convergente $(u_n)_{n \in \N}$ est unique. Pour $\forall (\ell_1, \ell_2) \in \R^2$
    \begin{align*}
        \text{Si } \lim_{n \to +\infty} u_n = \ell_1 \text{ et }\lim_{n \to +\infty} u_n = \ell_2 \text{ alors } \ell_1 = \ell_2
    \end{align*}
\end{theoreme}
\end{graybox}


\begin{proof}
         \par Procédons à un raisonnement par l'absurde. \\
         \par \noindent On suppose que $\ell_1 \neq \ell_2$. Posons $\varepsilon = \frac{1}{3} |l_1 - l_2| > 0$. \\
         D'après la définition de limite :
         \begin{align*}
             &\exists N_1 \in \N, \forall n \geq N_1, |u_n - \ell_1| \leq \varepsilon \\
             &\exists N_2 \in \N, \forall n \geq N_2, |u_n - \ell_2| \leq \varepsilon
         \end{align*}
         \par \noindent Posons $N = \max(N_1, N_2)$ Si $n \geq N$ alors :
         \begin{align*}
             &|u_n - \ell_1| \leq \varepsilon \\
             &|u_n - \ell_2| \leq \varepsilon
         \end{align*}
         \par \noindent Alors $|\ell_1 - \ell_2| \leq |u_n - \ell_1| + |u_n + \ell_2| \leq \varepsilon + \varepsilon = 2\varepsilon = \frac{2}{3}|\ell_1 - \ell_2|$.
    On en déduit $\frac{1}{3}|\ell_1 - \ell_2| \leq 0$, ce qui est absurde. Ainsi, on a montré que $\ell_1 = \ell_2$
\end{proof}

\begin{graybox}
    \begin{theoreme}[]
    Toute suite convergente est bornée.
\end{theoreme}
\end{graybox}

\begin{proof}
        \par \noindent Supposons qu'une suite $(u_n)_{n \geq 0}$ converge vers $\ell \in \R$. 
        \par \noindent On applique la définition de convergence avec $\varepsilon = 1$
        \begin{align*}
            \exists N \in \N, \forall n \geq N, |u_n - \ell| \leq 1 \vee \ell - 1 \leq u_n \leq \ell + 1
        \end{align*}
        \par \noindent Posons $M = \max(u_0, u_1, \ldots, u_{N - 1}, \ell + 1)$ et $m = \min(u_0, u_1, \ldots, u_{N -1}, \ell - 1)$.
        \begin{align*}
            &\forall n \in \N \\
            &\textnormal{Si } n < N, m \leq u_n \leq M \\
            &\textnormal{Si } n > N, m \leq \ell - 1 \leq u_n \leq \ell + 1 \leq M
        \end{align*}
        $(u_n)_{n \geq 0}$ est majorée par $M$ et minorée par $m$, elle est donc bornée.
    \end{proof}

\begin{remarque}
    La réciproque n'est pas vraie, il existe des suites bornées non convergentes comme par exemple : $u_n = (-1)^{n}$
\end{remarque}

\subsection{Opérations sur les suites convergentes}

\begin{graybox}
    \begin{theoreme}[]
    \par \noindent Soient $(u_n)_{n \geq 0}$ et $(v_n)_{n \geq 0}$ deux suites convergentes. \\
    On suppose que
    \begin{align*}
        \lim_{n \to +\infty} u_n = \ell_1, \lim_{n \to +\infty} v_n = \ell_2, (\ell_1, \ell_2) \in \R^2
    \end{align*}
    Pour $(\alpha, \beta) \in \R^2$, $(\alpha u_n + \beta v_n)_{n \geq 0}$ converge vers $\alpha \ell_1 + \beta \ell_2$.
\end{theoreme}
\end{graybox}

\begin{proof}
        Soit $\varepsilon > 0$ \\
        Posons : 
        \begin{align*}
            \varepsilon ' = \frac{\varepsilon}{|\alpha| + |\beta|} > 0, \textnormal{ avec } |\alpha| + |\beta| > 0
        \end{align*}
        \par \noindent Par définition de la limite :
        \begin{align*}
            &\exists N_1 \in \N, \forall n \geq N_1, |u_n - \ell_1| \leq \varepsilon' \\
            &\exists N_2 \in \N, \forall n \geq N_2, |v_n - \ell_2| \leq \varepsilon'
        \end{align*}
        Posons $N = \max(N_1, N_2)$. Si $n \geq N$, alors :
        \begin{align*}
            |u_n - \ell_1| \leq \varepsilon' \textnormal{ et } |v_n - \ell_2| \leq \varepsilon'
        \end{align*}
        Alors
        \begin{align*}
        |\alpha u_n + \beta v_n - (\alpha \ell_1 + \beta \ell_2)| &= |\alpha(u_n - \ell_1) + \beta(v_n - \ell_2)| \\
        & \leq |\alpha| |u_n - \ell_1| + |\beta| |v_n - \ell_2| \\
        &\leq (|\alpha| + |\beta|)\varepsilon' = \varepsilon 
        \end{align*}
        On a montré que $\alpha u_n + \beta v_n$ converge vers $\alpha \ell_1 + \beta \ell_2$.
    \end{proof}

\begin{remarque}
    En particulier : 
    \begin{align*}
        (u_n)_{n \geq 0} \textnormal{ convergente et } (v_n)_{n \geq 0} \textnormal{ convergente } \implies (u_n + v_n)_{n \geq 0} \textnormal{ convergente} 
    \end{align*}
    La réciproque est fausse : \\
    $u_n = n \textnormal{ divergente}$ et $v_n = -n \textnormal{ divergente}$ mais $u_n + v_n = 0$ convergente.
\end{remarque}

\begin{graybox}
    \begin{theoreme}[]
    Soient $(u_n)_{n \geq 0}$ et $(v_n)_{n \geq 0}$ deux suites convergentes. On suppose que :
    \begin{align*}
        \lim_{n \to +\infty} u_n = \ell_1, \lim_{n \to +\infty} v_n = \ell_2
    \end{align*}
    Alors la suite $(u_n v_n)_{n \geq 0}$ converge vers $\ell_1 \ell_2$
\end{theoreme}
\end{graybox}

\begin{proof}
        Comme $(u_n)_{n \geq 0}$ converge, elle est bornée. 
        \begin{align*}
            \exists M \in \R, \forall n \in \N, |u_n| \leq M
        \end{align*}
        Soit $\varepsilon > 0$. \\
        Posons
        \begin{align*}
            \varepsilon' = \frac{\varepsilon}{M + |\ell_2|} > 0
        \end{align*}
        Par définition de la limite 
        \begin{align*}
            &\exists N_1, \forall n \geq N_1, |u_n - \ell_1| \leq \varepsilon' \\
            &\exists N_2, \forall n \geq N_2, |v_n - \ell_2| \leq \varepsilon'  
        \end{align*}
        Posons $N = \max(N_1, N_2)$. Si $n \geq N$ alors 
        \begin{align*}
            |u_n - \ell_1| \leq \varepsilon' \textnormal{ et } |v_n - \ell_2| \leq \varepsilon'
        \end{align*}
        Par le calcul préliminaire :
        \begin{align*}
            |u_n v_n - \ell_1 \ell_2| < \varepsilon' (M + |\ell_2|) = \varepsilon
        \end{align*}
    \end{proof}

\begin{graybox}
    \begin{theoreme}[]
    Soit $(u_n)_{n \geq 0}$ une suite qui ne s'annule pas et qui converge vers $\ell \neq 0$. Alors la suite $\left(\frac{1}{u_n}\right)_{n \geq 0}$ converge vers $\frac{1}{\ell}$
\end{theoreme} 
\end{graybox}

\begin{proof}
        Soit $\varepsilon > 0$ \\
        On pose $\varepsilon' = \frac{|\ell|^2}{2} \varepsilon > 0$ \\
        Comme $u_n \xrightarrow[n \to +\infty]{} \ell, \exists N_1, \forall n \geq N_1, |u_n - \ell| \leq \varepsilon$\\
        Par ailleurs, posons $\varepsilon'' = \frac{|\ell|}{2} > 0$ \\
        Comme $u_n \xrightarrow[n  \to + \infty]{} \ell, \exists N_2, \forall n \geq N_2, |u_n - \ell| \leq \varepsilon''$ implique $|u_n| \geq |\ell| - |u_n - \ell| \geq |\ell| - \frac{|\ell|}{2} = \frac{|\ell|}{2}$ \\
        Soit $N = \max(N_1, N_2)$ \\
        Si $n \geq N, |u_n - \ell| \leq \varepsilon'$ et $|u_n| \geq \frac{|\ell|}{2}$, alors : 
        \begin{align*}
            \left| \frac{1}{u_n} - \frac{1}{\ell} \right| = \frac{|u_n - \ell|}{|u_n| - |\ell|} \leq \frac{\varepsilon'}{|\ell|\frac{|\ell|}{2}} = \frac{2 \varepsilon'}{|\ell|^2} = \varepsilon
        \end{align*}
        On a montré que $\left( \frac{1}{u_n} \right)_{n \geq 0}$ converge vers $\frac{1}{\ell}$
\end{proof}

\begin{remarque}~
    \begin{itemize}
    \item On peut avoir $u_n > 0$ est $\displaystyle{\lim_{n \to +\infty} u_n = 0}$ exemple : $u_n = \frac{1}{n}$. 
    \item Si $\displaystyle{\lim_{n \to +\infty} u_n > 0}$ alors $u_n > 0$ à partir d'un certain rang
    \end{itemize}
\end{remarque}

\subsection{Suites et inégalités}

\par On peut passer à la limite dans les inégalités larges.

\begin{graybox}
    \begin{theoreme}[]
    Soient $(u_n)_{n \geq 0}$ et $(v_n)_{n \geq 0}$ deux suites convergentes et telle que $\forall n \in \N, u_n \leq v_n$. Alors
    \begin{align*}
        \lim_{n \to +\infty} u_n \leq \lim_{n \to +\infty} v_n
    \end{align*}
\end{theoreme}
\end{graybox}

\begin{proof}
    Posons 
    \begin{align*}
        \ell_1 = \lim_{n \to +\infty} u_n \\
        \ell_2 = \lim_{n \to +\infty} v_n
    \end{align*}
    On veut montrer $\ell_1 \leq \ell_2$. 
    On raisonne par l'absurde en supposant que $\ell_2 < \ell_1$.
    \\
    Posons :
    \begin{align*}
        \varepsilon = \frac{\ell_1 - \ell_2}{3} > 0
    \end{align*}
    On a $\ell_2 + \varepsilon < \ell_1 - \varepsilon$.
    \\
    Comme $u_n \xrightarrow[]{} \ell_1, \exists N_1, \forall n \geq N_1, |u_n - \ell_1| \leq \varepsilon$.
    \\
    Comme $v_n \xrightarrow[]{} \ell_2, \exists N_2, \forall n \geq N_2, |v_n - \ell_2| \leq \varepsilon$.
    \\
    Soit $n \leq \max(N_1, N_2)$, alors :
    \begin{align*}
        &|u_n - \ell_1| \leq \varepsilon \textnormal{ donc } u_n \geq \ell_1 - \varepsilon \\
        &|v_n - \ell_2| \leq \varepsilon \textnormal{ donc } v_n \leq \ell_2 + \varepsilon
    \end{align*}
    Alors $v_n \leq \ell_2 + \varepsilon < \ell_1 - \varepsilon \leq u_n$ donc $v_n < u_n$ il y a donc une contradiction.
\end{proof}

\begin{remarque}
    Les inégalités strictes deviennent larges à la limite.
\end{remarque}

\begin{exemple}
    $n \geq 1, u_n = \frac{1}{2n}, v_n = \frac{1}{n}, \forall n \in \N, u_n < v_n$ et on a : 
    \begin{align*}
        \lim_{n \to +\infty} u_n = 0 = \lim_{n \to +\infty} v_n
    \end{align*}
\end{exemple}

\begin{graybox}
    \begin{corollaire}[]
    Si $(u_n)_{n \geq 0}$ converge vers $\ell$
    \begin{enumerate}
        \item Si $\forall n \in \N, u_n \leq M$, alors $\ell \leq M$
        \item Si $\forall n \in \N, u_n \geq m$, alors $\ell \geq m$
    \end{enumerate}
\end{corollaire}
\end{graybox}

\begin{proof}
    On applique le théorème au cas où une des suites est constante.
\end{proof}

\begin{theoreme}[Théorème des gendarmes]
    Soient $(u_n)_{n \geq 0}, (v_n)_{n \geq 0}, (w_n)_{n \geq 0}$ des suites. \\
    On suppose que 
    \begin{enumerate}
        \item $\forall n \in \N, u_n \leq v_n \leq w_n$
        \item $\displaystyle{\lim_{n \to +\infty} u_n = \lim_{n \to + \infty} v_n = \ell}$
    \end{enumerate}
    Alors $\displaystyle{\lim_{n \to +\infty} v_n = \ell}$
\end{theoreme}

\begin{proof}
    Soit $\varepsilon > 0$. \\
    $\exists N_1, \forall n \geq N_1, \ell - \varepsilon \leq u_n \leq \ell + \varepsilon$\\
    $\exists N_2, \forall n \geq N_2, \ell - \varepsilon \leq w_n \leq \ell + \varepsilon$ \\
    Soit $N = \max{(N_1, N_2)}$ \\
    Si $n \geq N$ $\ell - \varepsilon \leq v_n \leq \ell + \varepsilon$
\end{proof}

\begin{exemple}
    Soit $v_n = \frac{\sin{n}}{n}$ pour $n \geq 1$.
    On a pour tout $n \in \N^*$ :
    \begin{align*}
        -\frac{1}{n} \leq \frac{\sin{n}}{n} \leq \frac{1}{n}
    \end{align*}
    Comme $\displaystyle{\lim_{n \to +\infty} -\frac{1}{n} = \frac{1}{n} = 0}$.
    \\
    Par le théorème des gendarmes, on en conclut que $\displaystyle{\lim_{n \to +\infty} \frac{\sin{n}}{n} = 0}$
\end{exemple}

\subsection{Convergence et monotonie}
Il existe : 
\begin{itemize}
    \item des suites monotones non convergentes
    \begin{exemple}
        $u_n = n$
    \end{exemple}
    
    \item des suites convergentes non monotones
    \begin{exemple}
        $v_n = \frac{(-1)^n}{n}$
    \end{exemple}
\end{itemize}

\begin{graybox}
    \begin{theoreme}[]
    Toute suite croissante majorée converge. 
    \\
    De même, toute suite décroissante minorée converge.
\end{theoreme}
\end{graybox}

\begin{remarque}
    Soit $(u_n)_{n \geq 0}$ une suite croissante majorée. 
    La preuve (hors-programme) consiste à montrer que $(u_n)_{n \geq 0}$ converge vers la borne supérieure : $\underbrace{\forall n \in \N, u_n \leq v_n \leq w_n.}_{< +\infty \textnormal{ car }(u_n)_{n \geq 0} \textnormal{ est majorée }}$
\end{remarque}

\begin{graybox}
    \begin{theoreme}[Théorème des suites adjacentes]
    Soient $(u_n)_{n \geq 0}$ et $(v_n)_{n \geq 0}$ deux suites telles que :
    \begin{enumerate}
        \item $(u_n)_{n \geq 0}$ est croissante
        \item $(v_n)_{n \geq 0}$ est décroissante
        \item $(v_n - u_n)_{n \geq 0}$ converge vers 0.
    \end{enumerate}
    Alors $(u_n)_{n \geq 0} \textnormal{ et } (v_n)_{n \geq 0}$ convergent vers la même limite $\ell$ et $\forall n \in \N, u_n \leq \ell \leq v_n$.
\end{theoreme}
\end{graybox}

\begin{proof}
    Posons $w_n = v_n - u_n$. 
    \\
    La suite $(w_n)_{n \geq 0}$ tend vers 0 et est décroissante.
    \\
    Ceci implique $\forall n \in \N, w_n \geq 0$, c'est-à-dire $u_n \leq v_n$.
    \\
    On a $\forall n \in \N, u_0 \leq u_n \leq v_n \leq v_0$.
    \\
    La suite $(u_n)_{n \geq 0}$ est croissante et majorée par $v_0$, donc elle converge.
    \\
    La suite $(v_n)_{n \geq 0}$ est décroissante et minorée par $u_0$, donc elle converge.
    \\
    Posons : 
    \begin{align*}
        \lim_{n \to +\infty} u_n &= \ell_1 & \lim_{n \to +\infty} v_n &= \ell_2
    \end{align*}
    On a $\displaystyle{\lim_{n \to +\infty} w_n = 0 = \ell_2 - \ell_1}$, alors $\ell_2 = \ell_1$
    \\
    $\forall n \in \N$, on a :
    \begin{itemize}
        \item $u_n \leq \ell_1$ car $(u_n)$ est croissante
        \item $v_n \geq \ell_2$ car $(v_n)$ est décroissante
    \end{itemize}
\end{proof}

\begin{remarque}
    Si on suppose de plus que $(u_n)_{n \geq 0}$ et $(v_n)_{n \geq 0}$ convergent, on peut écrire :
    \begin{align*}
        &\lim_{n \to +\infty} (v_n - u_n) = \lim_{n \to +\infty} v_n - \lim_{n \to +\infty} u_n = 0 \\
        &\lim_{n \to +\infty} u_n = \lim_{n \to +\infty} v_n 
    \end{align*}
\end{remarque}

\begin{exemple}
    $u_n = n$ et $v_n = n + \frac{1}{n}$
    \begin{enumerate}
        \item $(u_n)_{n \geq 0}$ est croissante
        \item $(v_n)_{n \geq 0}$ n'est pas décroissante
        \item $\lim_{n \to +\infty} (v_n - u_n) = 0$
    \end{enumerate}
    Mais on ne peut pas écrire $\displaystyle{\lim_{n \to +\infty} u_n}$
\end{exemple}

\begin{exemple}
    Moyenne arithmético-géométrique
    \\
    Soit $a, b \geq 0$, $\frac{a + b}{2}$ moyenne arithmétique, $\sqrt{ab}$ moyenne géométrique. 
    \\
    On a $\forall a, b \geq 0, \sqrt{ab} \leq \frac{a + b}{2}$.
    \\
    En effet : 
    \begin{align*}
    &(\sqrt{a} - \sqrt{b})^2 \geq 0 \\
    &a + b - 2\sqrt{ab} \geq 0 \\
    &a + b \geq 2\sqrt{ab}
    \end{align*}

    \begin{equation}\label{moyennes}
        \frac{a + b}{2} \geq \sqrt{ab}
    \end{equation}
    
    Supposons $a \leq b$ et définissons deux suites $(u_n)_{n \geq 0}$ et $(v_n)_{n \geq 0}$ par $u_0 = a$ et $v_0 = b$. \\
    \begin{align*}
    \forall n \in \N,  u_{n + 1} &= \sqrt{u_n v_n} & \ v_{n + 1} &= \frac{u_n + v_n}{2}
    \end{align*}
    Montrons que $(u_n)_{n \geq 0}$ et $(v_n)_{n \geq 0}$ convergent vers la même limite à l'aide du théorème des suites adjacentes.
    Remarquons que $\forall n \in \N$
    \begin{align*}
        u_n \leq v_n
    \end{align*}
    Cela est vrai si $n = 0$ car $a < b$
    Si $n > 0$ en appliquant (\ref{moyennes}) a $u_{n-1}$ et $v_{n-1}$ \\
    Vérifions que $u_n$ est croissante, puis que $v_n$ est croissante et que $(v_n - u_n) \xrightarrow[n \to +\infty]{}  0$
    \begin{enumerate}
        \item Soit $n \in \N$
        \begin{align*}
            u_{n+1} - u_n &= \sqrt{u_nv_n} - u_n \\
                          &= \sqrt{u_n} \left( \sqrt{v_n} - \sqrt{u_n} \right) \geq 0
        \end{align*}
        Car $v_n \geq u_n$. $(u_n)$ est donc croissante
        \item Soit $n \in \N$
        \begin{align*}
            v_{n+1} - v_n &= \frac{u_n + v_n}{2} - v_n \\
                          &= \frac{u_n - v_n}{2} \leq 0
        \end{align*}
        Car $v_n \geq u_n$. $(v_n)$ est donc décroissante
        \item 
        Soit $n \in \N$
        \begin{align*}
            v_{n+1} - u_{n+1} &\leq v_{n+1} - u_n \text{ car } u_n \leq u_{n+1} \\
                              &\leq \frac{u_n+v_n}{2} - u_n \\
                              &\leq \frac{v_n - u_n}{2}
        \end{align*}
        Montrons par récurrence sur $n$ la propriété 
        \begin{equation*}
            P_n = "v_n - u_n \leq \frac{v_0 - u_0}{2^n}"
        \end{equation*}
        \begin{itemize}
            \item $P_0$ est vraie
            \item Supposons que $P_n$ vraie pour un entier n.
            Alors :
            \begin{align*}
                v_{n+1} - u_{n+1} \leq \frac{v_n - u_n}{2} \leq \frac{\frac{v_0 - u_0}{2^n}}{2} = \frac{v_0 - u_0}{2^{n+1}}
            \end{align*}
        \end{itemize}

        On a donc, $\forall n \in \N$
        \begin{align*}
            0 \leq v_n u_n \leq \frac{v_0 - u_0}{2} = \frac{b - a}{2}
        \end{align*}
        Comme 
        \begin{align*}
            \lim_{n \to +\infty} 0 = \lim_{n \to +\infty} \frac{b - a}{2} = 0
        \end{align*}
        par le théorème des gendarmes, on a :
        \begin{equation*}
            (v_n - u_n) \xrightarrow[n \to +\infty]{} 0
        \end{equation*}
        Par le théorème des suites adjacentes, $(u_n)$ et $(v_n)$ convergent vers la même limite.
    \end{enumerate}
\end{exemple}
\clearpage
\section{Suites extraites}
\textbf{Principe :} On part d'une suite $(u_n)_{n \geq 0}$ et on ne garde que certains des termes (en nombre infini) pour former une nouvelle suite qu'on appelle suite extraite de $(u_n)_{n\geq 0}$

\begin{exemple}~ 
    \begin{itemize}
        \item $(u_{2n})_{n \geq 0}$ est une sous-suite/suite-extraite de $(u_n)_{n \geq 0}$
        \item $(u_{2n+1})_{n \geq 0}$ aussi
        \item $(u_{3^n})_{n \geq 0}$ l'est également
    \end{itemize}
\end{exemple}

\begin{graybox}
    \begin{definition}[Extraction]
Une \textbf{extraction} est une fonction $\varphi : \N \to \N$ qui est strictement croissante. 
\end{definition}
\end{graybox}

\begin{graybox}
    \begin{definition}[Suite extraite]
Une \textbf{suite extraite} ou une \textbf{sous-suite} d'une suite $(u_n)_{n \geq 0}$ est une suite de la forme $(u_{\varphi(n)})_{n \geq 0}$ où $\varphi$ est une extraction. 
\end{definition}
\end{graybox}

\begin{graybox}
    \begin{proposition}[Propriétés]
    Soit $(u_{\varphi(n)})_{n \geq 0}$ une sous-suite de $(u_n)_{n \geq 0}$
    \begin{itemize}
        \item Si $(u_n)_{n \geq 0}$ est croissante, alors $(u_{\varphi(n)})_{n \geq 0}$ aussi.
        \item Si $(u_n)_{n \geq 0}$ est décroissante, alors $(u_{\varphi(n)})_{n \geq 0}$ aussi.
        \item Si $(u_n)_{n \geq 0}$ est majorée, alors $(u_{\varphi(n)})_{n \geq 0}$ aussi.
        \item Si $(u_n)_{n \geq 0}$ est minorée, alors $(u_{\varphi(n)})_{n \geq 0}$ aussi.
        \item Si $(u_n)_{n \geq 0}$ est converge vers $\ell$, alors $(u_{\varphi(n)})_{n \geq 0}$ aussi.
    \end{itemize}
\end{proposition} 
\end{graybox}

\begin{remarque}[Important]
    Même si $(u_n)$ ne converge pas, il se peut que des sous-suites convergent.
\end{remarque}

\begin{exemple}
    $u_n = (-1)^n$
    \begin{itemize}
        \item $u_{2n} = 1$ donc la sous-suite $(u_{2n})_{n \geq 0}$ converge vers 1
        \item $u_{2n+1} = -1$ donc la sous-suite $(u_{2n+1})$ converge vers -1
    \end{itemize}
    Mais $(u_n)$ ne converge pas car si elle convergeait vers $\ell \in \R$ on aurait
    \begin{align*}
        u_{2n} &\xrightarrow[]{} \ell \text{ donc } \ell = 1 \\
        u_{2n+1} &\xrightarrow[]{} \ell \text{ donc } \ell = -1
    \end{align*}
    Ce serait absurde.
\end{exemple}

\begin{graybox}
    \begin{proposition}
    Soit $(u_n)_{n \geq 0}$ une suite, alors :
    \begin{align*}
        (u_n)_{n \geq 0} \text{ converge } \iff (u_{2n})_{n \geq 0} \text{ et } (u_{2n+1})_{n \geq 0} \text{ convergent vers la même limite}
    \end{align*}
\end{proposition}
\end{graybox}

\begin{graybox}
    \begin{theoreme}[Théorème de Ramsey]
    Toute suite admet une sous-suite monotone.
\end{theoreme}
\end{graybox}

\begin{proof}
    Soit $(u_n)_{n \geq 0}$ une suite. Soit $E = \left\{ n \in \N, \forall m \geq n, u_m \leq u_n \right\}$ \\
    \underline{1er cas} \\
    $E$ est fini, donc majoré par un entier N, $\forall n \leq N, n \notin E$ donc $\exists m > n, u_m > u_n$(*)
    On définit alors par récurrence une extraction $\varphi \colon \N \to \N$
    en posant $\varphi(0) = N + 1$, puis, étant donnés $\varphi(0) < \varphi(1) < \cdots < \varphi(K)$, on choisit $\varphi(K+1)$ (possible par (*)) tel que 
    $u_{\varphi(K+1)} > u_{\varphi(K)}$ et la suite extraite $(u_{\varphi(n)})$ est croissante. \\
    \underline{2e cas} \\
    $E$ est infini. On pose alors $E = \left\{ \varphi(n) : n \in \N \right\}$ avec $\varphi \colon \N \to \N$
    
    \noindent $\forall k \in \N, \varphi(k) \in E$ Comme $\varphi(K+1) > \varphi(K)$, on a (par définition de $E$)
    \begin{align*}
        u_{\varphi(K+1)} \leq u_{\varphi(K)}
    \end{align*}
    et la sous-suite $(u_{\varphi(n)})_{n \in \N}$ est décroissante 
\end{proof}

\begin{graybox}
    \begin{theoreme}[Théorème de Bolzano-Weierstrass]
    Toute suite bornée admet une sous-suite convergente.
\end{theoreme}
\end{graybox}

\begin{proof}
    Soit $(u_n)_{n \geq 0}$ une suite bornée. D'après le théorème de Ramsey, $(u_n)_{n \geq 0}$ il existe une sous-suite monotone $(u_{\varphi(n)})_{n \geq 0}$. Comme $(u_{\varphi(n)})_{n \geq 0}$ est monotone et bornée, elle converge.
\end{proof}

\begin{exemple}~ 
\begin{itemize}
    \item $u_n = (-1)^n$
    On a $(u_{2n})$ et $(u_{2n+1})$ qui convergent
    \item $u_n = \cos(n)$ par le théorème de Bolzano-Weierstrass, elle admet des sous-suites convergentes, mais pas aussi simples à définir
\end{itemize}
    
\end{exemple}

\subsection{Limites infinies}
\begin{graybox}
    \begin{definition}
    Soit $(u_n)_{n \geq 0}$ une suite. On dit que $(u_n)_{n \geq 0}$ tend vers l'infini, si 
    \begin{align*}
    \forall A \in \R, \exists N \in \N, \forall n \geq N, u_n \geq A
    \end{align*}
    On écrit alors 
    \begin{align*}
    \lim_{n \to +\infty} u_n = +\infty \text{ ou } u_n \xrightarrow[n \to +\infty]{} +\infty
    \end{align*}
\end{definition}
\end{graybox}

\begin{remarque}
    Ne pas utiliser le mot "converger" pour une limite infinie. Il est réservé aux limites finies. On parlera de "diverger" vers l'infini.
\end{remarque}

\begin{graybox}
    \begin{definition}
    On dit que $(u_n)_{n \geq 0}$ tend vers $-\infty$ si 
    \begin{align*}
        \forall A \in R, \exists N \in N, \forall n \geq N, u_n \leq A
    \end{align*}
\end{definition}
\end{graybox}

\begin{remarque}
 Si $(u_n)_{n \geq 0}$ tend vers l'infini, toute sous-suite aussi   
\end{remarque}

\begin{graybox}
\begin{theoreme}
    Soit $(u_n)_{n \geq 0}$ une suite croissante. Alors :
    \begin{itemize}
        \item ou bien $(u_n)_{n \geq 0}$ converge (vers une limite finie)
        \item ou bien $(u_n)_{n \geq 0}$ tend vers $+\infty$
    \end{itemize}
\end{theoreme}
\end{graybox}
\begin{proof}~ 
    \begin{itemize}
        \item Si $(u_n)$ est majorée, elle converge d'après le théorème de convergence monotone car elle est croissante et majorée.
        \item Si $(u_n)$ n'est pas majorée, montrons qu'elle tend vers $+\infty$ \\
        Soit $A$ un réel. \\
        Comme $(u_n)$ n'est pas majorée, 
        \begin{align*}
            \exists N \in \N, u_N \geq A \\
            \forall n \geq N, u_n \geq u_N \geq A
        \end{align*}
        On a montré que :
        \begin{align*}
            u_n \xrightarrow[n \to +\infty]{} +\infty
        \end{align*}
    \end{itemize}
\end{proof}

\begin{exemple}
    Un autre exemple de suite : $u_n = n(-1)^n$. \\
    La suite $(u_n)_{n \geq 0}$ n'a pas de limite finie ou infinie.
    \begin{align*}
        u_{2n} \xrightarrow[n \to +\infty]{} +\infty,\ u_{2n+1} \xrightarrow[n \to +\infty]{} -\infty
    \end{align*}
\end{exemple}

\begin{graybox}
\begin{theoreme}[Théorèmes de comparaison]~ 
    \begin{itemize}
        \item Si $(u_n)_{n \in \N}$ et $(v_n)_{n \in \N}$ vérifient $\forall n \in \N, u_n \leq v_n$ 
            \begin{align*}
                \lim_{n \to +\infty} u_n = +\infty \implies \lim_{n \to +\infty} v_n = +\infty
            \end{align*}
        \item Si $(u_n)_{n \in \N}$ et $(v_n)_{n \in \N}$ vérifient $\forall n \in \N, u_n \leq v_n$
            \begin{align*}
                \lim_{n \to +\infty} v_n = -\infty \implies \lim_{n \to +\infty} u_n = -\infty
            \end{align*}
    \end{itemize}
\end{theoreme}
\end{graybox}
\begin{remarque}[Formes indéterminées des sommes de limites]
Les résultats pour les limites finies ne s'étendent pas tous aux limites infinies.
\begin{table}[!h]
\centering
\begin{tabular}{c c}
HYPOTHESES & CONCLUSION \\
\hline
$\lim u_n$ et  $\lim v_n$ & $\lim (u_n + v_n)$ \\
$\ell \in \R$ et $\ell ' \in \R$ & $\ell + \ell'$ \\
$\ell \in \R$ et $+\infty$ & $+\infty$ \\
$\ell \in \R$ et $-\infty$ & $-\infty$ \\
$+\infty$ et $+\infty$ & $+\infty$ \\
$-\infty$ et $-\infty$ & $-\infty$ \\
$-\infty$ et $+\infty$ & FI \\
\hline
\end{tabular}
\caption{Formes indéterminées pour les sommes de limites}
\end{table}
\\
Quand on a une forme indéterminée, la suite concernée peut avoir tous les comportements possibles.
\end{remarque}

\begin{exemple}
$u_n = n$
\begin{itemize}
    \item $v_n = n + 3$ 
    \item $v_n' = n + \sqrt{n}$
    \item $v_n'' = n + (-1)^{n}$
\end{itemize}
On a :
\begin{align*}
    \lim_{n \to +\infty} u_n = \lim_{n \to +\infty} v_n = \lim_{n \to +\infty} v_n' = \lim_{n \to +\infty} v_n'' = +\infty
\end{align*}
\begin{align*}
    v_n - u_n, \ v_n' - u_n, \ v_n'' - u_n \text{ sont des formes indéterminées}
\end{align*}
\begin{align*}
    v_n - u_n = 3 &\implies \lim_{n \to +\infty} = 3 \\
    v_n' - u_n = \sqrt{n} &\implies \lim_{n \to +\infty} = +\infty \\
    v_n'' -u_n = (-1)^n \text{ n'a pas de limite finie ou infinie}
\end{align*}
\end{exemple}

\begin{remarque}[Formes indéterminées des produits des limites]
    \begin{table}[!h]
        \centering
        \begin{tabular}{cc}
            HYPOTHESES & CONCLUSION \\
            \hline
            $\lim u_n$ et $\lim v_n$ & $\lim (u_n v_n)$ \\
            $\ell > 0$ et $+\infty$ & $+\infty$ \\
            $\ell < 0$ et $+\infty$ & $-\infty$ \\
            $0$ et $+\infty$ & FI \\
            $\ell > 0$ et $-\infty$ & $-\infty$ \\
            $\ell < 0$ et $+\infty$ & $-\infty$ \\
            $0$ et $-\infty$ & FI \\
            $+\infty$ et $+\infty$ & $+\infty$ \\
            $+\infty$ et $-\infty$ & $-\infty$ \\
            $-\infty$ et $-\infty$ & $+\infty$ \\
            \hline
        \end{tabular}
        \caption{Formes indéterminées des produits de limites}
    \end{table}
\end{remarque}

\begin{remarque}[Formes indéterminées des quotients de limites]
    Supposons que $\forall n \in \N, v_n \neq 0$
    \begin{table}[!h]
    \centering
    \begin{tabular}{cc}
        HYPOTHESES & CONCLUSION \\
        \hline 
        $\lim u_n$ et $\lim v_n$ & $\lim \frac{u_n}{v_n}$ \\
        $0$ et $0$ & FI \\
        $\pm \infty$ et $\pm \infty$ & FI \\
        $0$ et $\pm \infty$ & 0 \\
        $+\infty$ et $0$ & $\begin{cases} 
                             &+\infty \text{ si } \forall n, v_n > 0 \\
                             &-\infty \text{ si } \forall n, v_n < 0 
                            \end{cases}$ \\
        $-\infty$ et $0$ & $\begin{cases}
                                &-\infty \text{ si } \forall n, v_n > 0 \\
                                &+\infty \text{ si } \forall n, v_n < 0
                            \end{cases}$ \\
                            \hline
    \end{tabular}
    \caption{Formes indéterminées des quotients de limites}
\end{table}
\end{remarque}

\begin{remarque}
    Une autre forme indéterminée est $"1^{\infty}$, autrement dit, si :
    \begin{align*}
        \lim_{n \to +\infty} u_n = 1 \text{ et } \lim_{n \to +\infty} v_n = +\infty
    \end{align*}
    le comportement de $u_n^{v_n}$ est indéterminé. Un bon réflexe pour étudier ces suites est d'utiliser la formule 
    \begin{align*}
        a^b = \exp(b\ln a)
    \end{align*}
    On a :
    \begin{align*}
        u_n^{v_n} = \exp(v_n \ln u_n)
    \end{align*}
\end{remarque}

\begin{exemple}
On peut montrer que la suite :
\begin{align*} 
    \left(1 + \frac{1}{n}\right)^n \xrightarrow[n \to +\infty]{} e
\end{align*}
\end{exemple}

\begin{remarque}[Notations asymptotiques]
Soient $(u_n)$ et $(v_n)$ deux suites avec $\forall n \in \N$, $u_n > 0$ et $v_n > 0$ On dit que $(u_n)$ est négigeable devant $(v_n)$, ou 
\begin{align*}
    u_n = o(v_n), \text{ lorsque } \lim_{n \to +\infty} \frac{u_n}{v_n} = 0
\end{align*}
\end{remarque}

\begin{exemple}~
    \begin{itemize}
        \item $\sqrt{n} = o(n)$
        \item $\ln n = o(n^{\alpha})$ pour tout $\alpha > 0$
    \end{itemize}
\end{exemple}

\begin{remarque}
On dit que $u_n = O(v_n)$ si la suite $\frac{u_n}{v_n}$ est majorée
\end{remarque}

\begin{exemple}
    \begin{align*}
        3n^2 + 2n + 5 = O(n^2)
    \end{align*}
    Si $P$ est un polynôme de degré $d$
    \begin{align*}
        P(n) = O(n^d)
    \end{align*}
\end{exemple}

\begin{remarque}
Ces notations sont très utilisées en informatique
\begin{align*}
    u_n = o(v_n), v_n = O(w_n) \implies u_n = o(w_n) \\
    u_n = O(v_n), v_n = O(w_n) \implies u_n = O(w_n)
\end{align*}
\end{remarque}

\begin{graybox}
\begin{definition}[Suite de Cauchy]
C'est un moyen de parler de suites convergentes sans mentionner la limite. \\
Une suite $(u_n)$ est de Cauchy si :
\begin{align*}
    \forall \varepsilon > 0, \exists N \in \N, \forall n_1 \geq N, \forall n_2 \geq N, |u_{n_1} - u_{n_2}| \leq \varepsilon
\end{align*}
\end{definition}
\end{graybox}
