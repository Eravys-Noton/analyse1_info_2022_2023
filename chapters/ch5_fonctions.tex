\chapter{Continuité et limites de fonctions}
\section{Limites}
Soit $I$ un intervalle de $\R$ et $f \colon I \to \R$ une fonction.
Soit $x_0 \in I$ ou $x_0$ une borne de $I$ (possiblement $\pm \infty$).
On veut définir ce que veut dire 
\begin{align*}
    \lim_{x \to x_0} f(x) = \ell,\ \ell \in \R \text{ ou } \ell = \pm \infty
\end{align*}

\begin{graybox}
\begin{definition}[Limite d'une fonction en un point]~ 
\\
\textbf{1er cas} \\
$x_0 \in \R$, $\ell \in \R$ \\
On dit que $\displaystyle \lim_{x \to x_0} f(x) = \ell$
\begin{align*}
    \text{Si } \forall \varepsilon > 0, \exists \delta > 0 \text{ tel que si } |x -x_0| \leq \delta, \text{ alors } |f(x) - \ell| \leq \varepsilon
\end{align*}
\textbf{2e cas} \\
$x_0 \in \R$, $\ell = \pm \infty$ \\
On dit que $\displaystyle \lim_{x \to x_0} f(x) = +\infty$ 
\begin{align*}
    \text{Si } \forall A \in \R, \exists \delta > 0 \text{ tel que si } |x - x_0| \leq \delta \text{ alors } f(x) \geq A
\end{align*}
On dit que $\displaystyle \lim_{x \to x_0} f(x) = -\infty$ 
\begin{align*}
    \text{Si } \forall A \in \R, \exists \delta > 0 \text{ tel que si } |x - x_0| \leq \delta \text{ alors } f(x) \leq A
\end{align*}
\end{definition}
\end{graybox}

\begin{graybox}
    \begin{definition}[Limite d'une fonction en l'infini]~
        \\
\textbf{1er cas} \\
$x_0 = \pm \infty,\ \ell \in \R$ \\
On dit que $\displaystyle \lim_{x \to +\infty} f(x) = \ell $
\begin{align*}
    \text{Si } \forall \varepsilon > 0, \exists B \in \R \text{ tel que si } x \geq B \text{ alors } |f(x) - \ell| < \varepsilon
\end{align*}
On dit que $\displaystyle \lim_{x \to -\infty} f(x) = \ell $
\begin{align*}
    \text{ Si} \forall \varepsilon > 0, \exists B \in \R \text{ tel que si } x \leq B \text{ alors } |f(x) - \ell| \leq \varepsilon
\end{align*}
\textbf{2e cas} \\
$x_0 = \pm \infty$, $\ell = \pm \infty$ \\
On dit que $\displaystyle \lim_{x \to +\infty} = +\infty$ 
\begin{align*}
    \text{Si } \forall A \in \R, \exists B \in \R \text{ tel que si } x \geq B \text{ alors } f(x) \geq A
\end{align*}
On dit que $\lim_{x \to +\infty} = -\infty$
\begin{align*}
    \text{Si } \forall A \in \R, \exists B \in \R \text{ tel que si } x \geq B \text{ alors } f(x) \leq A
\end{align*}
On dit que $\displaystyle \lim_{x \to -\infty} f(x) = +\infty$ 
\begin{align*}
    \text{Si } \forall A \in \R, \exists B \in \R \text{ tel que si } x \leq B \text{ alors } f(x) \geq A
\end{align*}
on dit que $\displaystyle \lim_{x \to -\infty} f(x) = -\infty$
\begin{align*}
    \text{Si } \forall A \in \R, \exists B \in \R \text{ tel que si } x \leq B \text{ alors } f(x) \leq A
\end{align*}

\end{definition}
\end{graybox}
\begin{remarque}
    Lorsque $\displaystyle \lim_{x \to x_0} f(x) = 
        \begin{cases}
            +\infty \\
            \text{ou} \\
            -\infty
        \end{cases}$
    la droite verticale d'équation $x = x_0$ est une asymptote verticale au graphe de $f$
\end{remarque}

\begin{remarque}
    Lorsque $\displaystyle \lim_{x \to x_0} f(x) = \ell \text{ ou } \lim_{x \to x_0} f(x) = \ell$, la droite horizontale d'équation $y = \ell$ une asymptote horizontale au graphe de $f$
\end{remarque}

\begin{graybox}
    \begin{theoreme}[Caractérisation séquentielle des limites]
        Soit $I \subset \R$ un intervalle, $f \colon I \to \R$, $x_0 \in I$ ou une borne de $I$ (éventuellement $\pm \infty$), $\ell \in \R$ ou $\ell = +\infty$ ou $\ell = -\infty$.     
    \begin{align*}
        \text{Alors } \lim_{x \to x_0} f(x) = \ell \iff \text{Pour toute suite } (u_n)_{n \in \N} \text{ telle que } \lim_{n \to \infty} u_n = x_0, \text{ on a } \\ \lim_{n \to \infty} f(u_n) = \ell
    \end{align*}
    \end{theoreme}
\end{graybox}

\par Par conséquent, les théorèmes sur les limites de suites ont des analogues immédiats pour les limites de fonctions.

\begin{graybox}
\begin{theoreme}
Soit $I$ un intervalle de $\R$.. $f, g, h \colon I \to \R$, $x_0 \in I$ ou une borne de $I$.
\begin{enumerate}
\item Si 
$
\begin{cases}
\lim_{x \to x_0} f(x) = \ell_1 \\
\lim_{x \to x_0} g(x) = \ell_2
\end{cases}
$
alors $\forall a, b \in \R$
\begin{itemize}
\item $\lim_{x \to x_0} (af + bg) (x) = a \ell_1 + b \ell_2$
\item $\lim_{x \to x_0} f(x)g(x) = \ell_1 \ell_2$
\item Si $g$ ne s'annule pas et $\ell_2 \neq 0$ $\lim_{x \to x_0} \frac{f(x)}{g(x)} = \frac{\ell_1}{\ell_2}$
\end{itemize}
\item Si $\lim_{x \to x_0} f(x) = \ell_1, \lim_{x \to x_0}g(x) = \ell_2$ et $\forall x \in I, f(x) \leq g(x)$ alors $\ell_1 \leq \ell_2$
\item Si $\forall x \in I, f(x) \leq g(x) \leq h(x)$ et si $\lim_{x \to x_0} f(x) = \lim_{x \to x_0} h(x) = \ell$ alors $\lim_{x \to x_0} g(x) = \ell$
\end{enumerate}
\end{theoreme}
\end{graybox}

\begin{exemple}
Considérons la fonction 
\begin{align*}
g : 
\begin{cases}
\R^*_+ &\to \R \\
x &\mapsto \frac{\sin(x)}{x}
\end{cases}
\end{align*}
On a $\forall x > 0$
\begin{align*}
-1 \leq \sin(x) \leq 1 \\
\frac{-1}{x} \leq \frac{\sin(x)}{x} \leq \frac{1}{x}
\end{align*}
Comme 
\begin{align*}
\lim_{x \to +\infty} \frac{-1}{x} = \lim_{x \to +\infty} \frac{1}{x} = 0
\end{align*}
Par théorème des gendarmes 
\begin{align*}
\lim_{x \to +\infty} g(x) = 0
\end{align*}
\end{exemple}

\begin{graybox}
\begin{theoreme}[Composition des limites]
Soient $I, J$ deux intervalles et les fonctions $f\colon I \to J$, $g \colon J \to \R$, $x_0 \in I$ ou une borne de $I$. On suppose 
\begin{align*}
&\lim_{x \to x_0} f(x) = y \text{ avec } y \in I \text{ ou avec une borne de } J \\
&\lim_{y \to z} g(z) = \ell \text{ existe }
\end{align*}
Alors 
\begin{align*}
\lim_{x \to x_0} g(f(x)) = \ell
\end{align*} 
\end{theoreme}
\end{graybox}

\begin{exemple}
Soit 
\begin{align*}
h \colon 
\begin{cases}
\R^*_+ &\to \R \\
x &\mapsto \sqrt{\frac{3}{x} + 7}
\end{cases}
\end{align*}
On pose $h$ comme étant une composée avec :
\begin{align*}
f \colon 
&\begin{cases}
\R^*_+ &\to \R \\
x &\mapsto \frac{3}{x} + 7
\end{cases}
&
g \colon 
&\begin{cases}
\R^+ &\to \R^+ \\
x &\mapsto \sqrt{x}
\end{cases}
\end{align*}
On a 
\begin{align*}
&\lim_{x \to +\infty} f(x) &= 7 & \lim_{x \to 7} g(x) &= \sqrt{7} \implies \lim_{x \to +\infty} h(x) = \sqrt{7}
\end{align*}
\end{exemple}

\begin{graybox}
\begin{definition}[Limite à gauche / à droite]
Soit $I \subset \R$ un intervalle, $x_0 \in I$
\begin{align*}
f \colon I \to \R & \ell \in \R \\
\text{ ou } f \colon I\backslash\{x_0\} \to \R & \text{ ou } \ell = +\infty \text{ ou } \ell = -\infty
\end{align*}
On dit que $\ell$ est la limite à gauche de $f$ en $x_0$ 
\begin{align*}
\lim_{\substack{x \to x_0 \\  x < x_0}} f(x) = \ell \text{ ou } \lim_{x \to x_0^-} f(x) = \ell
\end{align*}
Si la limite de $f_{\mid I \cap ]-\infty, x_0]}$  en $x_0$ vaut $\ell$\\
Si $\ell \in \R$ cela équivaut à 
\begin{align*}
\forall \varepsilon > 0, \exists \delta > 0, x_0 - \delta < x < x_0 \implies |f(x) - \ell| \leq \varepsilon
\end{align*}
On dit que $\ell$ est la limite de $f$ en $x_0$,
\begin{align*}
\lim_{\substack{x \to x_0 \\ x > x_0}} f(x) = \ell \text{ ou } \lim_{x \to x_0^+} f(x) = \ell
\end{align*}
Si la limite de $f_{\mid I \cap ]x_0, +\infty]}$ en $x_0$ vaut $\ell$
\end{definition}
\end{graybox}

\begin{remarque}
$f \colon I \to \R$ et $J \subset I$ 
\\
$f_{\mid J} \colon 
\begin{cases}
J &\to \R \\
x &\mapsto f(x)
\end{cases}
$
est la restriction de $f$ à $J$.
\end{remarque}

\par \noindent Soit $I$ un intervalle, $x_0 \in I$ (pas une borne)
\begin{itemize}
\item Si $f \colon I \to \R$, alors 
\begin{align*}
\lim_{x \to x_0} f(x) = \ell \iff \lim_{x \to x_0 \newline x < x_0} f(x) = f(x_0) = \lim_{\substack{x \to x_0 \\ x > x_0}} f(x) = \ell
\end{align*}
\item Si $f \colon I \to \R$, alors 
\begin{align*}
\lim_{x \to x_0} f(x) = \ell \iff \lim_{x \to x_0 \newline x < x_0} f(x) = \lim_{\substack{x \to x_0 \\ x > x_0}} f(x) = \ell
\end{align*}
\end{itemize}

\section{Continuité}
\begin{graybox}
\begin{definition}
Soit $I$ un intervalle, $x_0 \in I$, $f \colon I \to \R$
\begin{itemize}
\item On dit que $f$ est continue en $x_0$ si 
\begin{align*}
\lim_{x \to x_0} f(x) = f(x_0)
\end{align*}
Avec les quantificateurs, on a :
\begin{align*}
f \text{ continue } \iff \forall \varepsilon > 0, \exists \delta > 0, si x \in I \text{ vérifie } |x - x_0 | \leq \delta \text{ alors } |f(x) - f(x_0| \leq \varepsilon
\end{align*}
\item On dit que $f$ est continue sur $I$ si elle est continue en tout point de I.
\begin{align*}
\forall x_0 \in I, \forall \varepsilon > 0, \exists \delta > 0, \text{ si } x \in I \text{ vérifie } |x - x_0| \leq \delta \text{ alors } |f(x) - f(x_0| \leq \varepsilon
\end{align*}
\item On dit que $f$ est continue à gauche en $x_0$ si 
\begin{align*}
\lim_{\substack{x \to x_0 \\ x < x_0}} f(x) = f(x_0)
\end{align*}
et continue à droite en $x_0$ si
\begin{align*}
\lim_{\substack{x \to x_0 \\ x > x_0}} f(x) = f(x_0)
\end{align*}
Ainsi $f$ est continue en $x_0$ si et seulement si elle est continue à gauche et à droite de $x_0$
\end{itemize}
\end{definition} 
\end{graybox}

\begin{exemple}
$E \colon \R \to \Z$ partie entière.
\begin{itemize}
\item Si $x_0 \in \R \backslash \Z$, $E$ est continue en $x_0$
\item Si $x_0 \in \Z$, $E$ est continue à droite en $x_0$ mais pas continue à gauche en $x_0$
\end{itemize}
\end{exemple}

\begin{exemple}
$ 
f \colon 
\begin{cases}
\R &\to \R \\
x &\mapsto 
\begin{cases}
1 \text{ si } x \in \Q \\
0 \text{ si } x \notin \Q
\end{cases}
\end{cases}
$
cette fonction n'est continue en aucun point.
\end{exemple}

\begin{remarque}
Les fonctions usuelles 
\begin{itemize}
\item $\exp$
\item $\ln$
\item $\cos$
\item $\sin$
\item $\tan$
\item $\arcsin$
\item $\arccos$
\item $\arctan$
\item $\cosh$
\item $\sinh$
\item $\tanh$
\item $\sqrt{x}$
\end{itemize}
sont continues dans leurs domaines de définition.
\end{remarque}

\begin{graybox}
\begin{theoreme}[Opérations sur les fonctions continues]
La somme, le produit, la composition, le quotient de deux fonctions continues est une fonction continue.
\end{theoreme}
\end{graybox}

\begin{exemple}
$
f \colon 
\begin{cases}
\R &\to \R \\
x &\mapsto \sin(x^2 - 3)\exp(2\cos(x - 1))
\end{cases}
$
est continue comme produit de fonctions continues.
\end{exemple}

\begin{graybox}
\begin{theoreme}[Théorème des valeurs intermédiaires]
Soit $I = [a, b]$ un intervalle de $\R$ avec $a < b$ et $f \colon I \to \R$ une fonction continue.
Soit $y \in \R$ tel que $f(a) \leq y \leq f(b)$ ou $f(b) \leq y \leq f(a)$.
Alors il existe un point $c \in [a, b]$ tel que $f(c) = y$
\end{theoreme}
\end{graybox}

\begin{proof}
On utilise la propriété de la borne supérieure de $\R$. Toute partie non vide majorée $A \subset \R$ admet une borne supérieure $\sup(A)$
\begin{itemize}
\item Quitte à changer $f$ en $-f$ et $y$ en $-y$ on peut supposer 
\begin{align*}
f(a) \leq y \leq f(b) 
\end{align*}
\item Soit $E = \{x \in I \text{ tel que } f(x) \leq y\}$,
$a \in E$ donc $E$ est non vide,
$E \subset I$ donc $E$ est majoré donc on peut poser $c = \sup(E)$ \\
Puisque $c = \sup(E)$, il existe une suite $(c_n)$ d'éléments de $E$ telle que 
$\lim_{n \to +\infty} c_n = c$ \\
Comme $f$ est continue, on a 
\begin{align*}
\lim_{n \to +\infty} f(c_n) = f(c)
\end{align*}
Puisque $c_n \in E$, on a $f(c_n) \leq y$. En passant à la limite, on a 
\begin{align*}
f(c) \leq y
\end{align*}
Il reste à montrer $f(c) \geq y$
\begin{itemize}
\item Si $c = b$, on a bien 
\begin{align*}
f(c) = f(b) \leq y
\end{align*}
\item Si $c < b$, pour $n$ assez grand, 
\begin{align*}
c < \underbrace{c + \frac{1}{n}}_{\substack{\text{pas dans } E,\\ \text{ car } c= \sup(E)}} \leq b
\end{align*}
\begin{align*}
c + \frac{1}{n} \notin E \implies f\left( c + \frac{1}{n} \right) > y
\end{align*}
On a $\lim_{n \to +\infty} c + \frac{1}{n} = c$ et $f$ étant continue
\begin{align*}
\lim_{n \to +\infty} f\left( c + \frac{1}{n} \right) = f(c)
\end{align*}
Comme $f\left(c + \frac{1}{n}\right) > y$, on en déduit en passant à la limite $f(c) \geq y$
\end{itemize}
\end{itemize}
\end{proof}

\begin{exemple}
$
f \colon 
\begin{cases}
\R &\to \R \\
x &\mapsto x^4 + x^2 - 6
\end{cases}
$
On veut montrer que $f$ s'annule sur $[0,2]$ $f$ est continue.
\\
\begin{align*}
f(0) &= -6 & f(2) &= 14
\end{align*}
Comme $-6 \leq 0 \leq 14$, par le TVI, il existe $c \in [0,2]$ tel que $f(c) = 0$
\end{exemple}

\begin{remarque}
Il est important que $I$ soit un intervalle pour utiliser le TVI.
\end{remarque}
\begin{exemple}
$
f \colon 
\begin{cases}
\R^* &\to \R \\
x &\mapsto \frac{1}{x}
\end{cases}
$
même si $f$ est continue sur $R^*$, $f(-1) = -1$ et $f(1) = 1$, il n'existe pas de $c \in [-1, 1]$ tel que $f(c) = 0$, le TVI ne s'applique pas car $f$ n'est pas définie sur $[-1, 1]$
\end{exemple}

\begin{exemple}
Soit 
$
f \colon 
\begin{cases}
\R &\to \R \\
x &\mapsto x^2 + \cos(x) -3
\end{cases}
$ 
\\
Montrer que $f$ s'annule sur $[-5, 5]$ $f(-5) = f(5) = 22 + \cos(5) > 0 $. \\
Mais comme $f(0) = -2$, on peut appliquer le TVI sur $[0, 5]$, $f(0) \leq 0 \leq f(5)$ et conclure que $\exists c \in [0, 5]$ tel que $f(c) = 0$.
\end{exemple}

\begin{remarque}
Le même énoncé serait faux pour $\Q$ à la place de $\R$. 
\end{remarque}

\begin{exemple}
$
f \colon 
\begin{cases}
\Q &\to \R \\
x &\mapsto x^2
\end{cases}
$
\begin{align*}
f(0) &= 0 & f(2) &= 4 \\
f(0) \leq 2 \leq f(2)
\end{align*}
Par le théorème des valeurs intermédiaires, il existe un réel $c \in [0, 2]$ tel que $f(c) = 2, c = \sqrt{2}$, mais il n'existe pas de rationnel $c$ te lque $f(c) = 2$
\end{exemple}

\begin{remarque}
L'image d'un intervalle par une fonction continue est un intervalle.
\end{remarque}

\begin{exemple}
$
f \colon
\begin{cases}
\R &\to \R \\
x &\mapsto x^2 + 1
\end{cases}
$
\begin{align*}
f([-1, 4]) =
\end{align*}
$f'(x) = 2x$ donc 
$
\begin{cases}
f'(x) \leq 0 \text{ si } x \in [-1, 0] \\
-f'(x) \geq 0 \text{ si } x \in [0, 4]
\end{cases}
$
\end{exemple}

\section{Limites, continuité et monotonie}
Les fonctions monotones ont des propriétés spécifiques 
\begin{graybox}
\begin{theoreme}[Théorème de la "limite monotone"]
Soit $I=]a, b[$ avec $a < b$ et $f \colon I \to \R$ croissante.
\begin{enumerate}
\item $f$ admet une limite en $b$, qui est finie si et seulement si $f$ est majorée.
\item $f$ admet une limite en $a$, qui est finie si et seulement si $f$ est minorée.
\item Pour tout $x_0 \in I$, $f$ a une limite à gauche et à droite en $x_0$
et 
\begin{align*}
\lim_{\substack{x \to x_0 \\ x < x_0}} f(x) \leq f(x_0) \leq \lim_{\substack{x \to x_0 \\ x > x_0}} f(x)
\end{align*}
\end{enumerate}
Enoncé analogue pour $f$ décroissante.
\end{theoreme}
\end{graybox}

\begin{graybox}
\begin{theoreme}[Stricte monotonie et bijectivité]
Soit $f \colon [a, b] \to \R$ continue
\begin{itemize}
\item Si $f$ est strictement croissante, $f \colon [a, b] \to [f(a), f(b)]$ est une bijection.
\item Si $f$ est strictement décroissante, $f \colon [a, b] \to [f(b), f(a)]$ est une bijection. 
Ce théorème a été utilisé pour définir par exemple $\arcsin$ car il montre que $\sin \colon \left[-\frac{\pi}{2}\right] \to [-1,1]$ est bijective.
En effet strictement monotone $\implies$ injective et on montre que $f$ est injective grâce au TVI.
\end{itemize}
\end{theoreme}
\end{graybox}

\begin{exemple}
$
f \colon 
\begin{cases}
\R &\to \R \\
x &\mapsto 2x^3 - 9x^2 + 12x - 1
\end{cases}
$
$f$ est continue, 
\begin{align*}
f'(x) = 6x^2 - 18x + 12 = 6(x^2 - 3x + 2) = 6(x-1)(x-2)
\end{align*}
\begin{align*}
f(1) &= 2 - 9 + 12 - 1 & f(2) &= 16 - 36 + 24 - 1  \\
&= 4 & &=3
\end{align*}
Si $x \neq 0$, $f(x) = x^3 \left( 2 - \frac{9}{x} + \frac{12}{x^2} - \frac{1}{x^3} \right)$ et donc 
\begin{align*}
\lim_{x \to +\infty} f(x) = +\infty \\
\lim_{x \to -\infty} f(x) = -\infty
\end{align*}
Comme $f$ est strictement décroissante sur $[1, 2]$, c'est une bijection de $[1, 2]$ sur $[3, 4]$. Le théorème est également vrai pour un intervalle $]a, b[$ avec $a \in \R \cup \{-\infty\}, b \in \R \cup \{+\infty\}$ si 
$f \colon ]a, b[ \to \R$ est continue et 
\begin{itemize}
\item strictement croissante, c'est une bijection de $]a, b[$ sur $]\lim_{x \to a} f(x), \lim_{x \to b} f(x)[$
\item strictement décroissante, c'est un bijection de $]a, b[$ sur $]\lim_{x \to b} f(x), \lim_{x \to a} f(x)[$
\end{itemize}
Dans l'exemple précédent, $f$ est une bijection de $]-\infty, 1[$ sur $]-\infty, 4[$.
\end{exemple}

\begin{exemple}
$
f \colon
\begin{cases}
]0, +\infty[ &\to \R \\
x &\mapsto \frac{1}{x^2 + x}
\end{cases}
$
$f$ est continue et strictement décroissante car la fonction $x \mapsto x^2 + x$ est strictement croissante sur $]0, +\infty[$
\begin{align*}
\lim_{x \to 0^+} f(x) &= +\infty & \lim_{x \to +\infty} f(x) &= 0
\end{align*}
donc $f$ est une bijection de $]0, +\infty[$ sur $]0, +\infty[$.
\end{exemple}

\begin{graybox}
\begin{theoreme}
Soit $I$ un intervalle de $\R$ et $f \colon I \to \R$ continue injective.
\\
Alors $f$ est strictement monotone, donc bijective.
\\
Si on pose $J = f(I)$ , la bijection réciproque $f^{-1} \colon J \to I$ est continue.
\end{theoreme}
\end{graybox}

\begin{remarque}
Dans le théorème précédent, $f^{-1}$ est aussi strictement monotone.
\end{remarque}

\par \noindent \textbf{Fonctions continues sur un segment}

\begin{remarque}
Segment = intervalle fermé borné.
\end{remarque}
\begin{graybox}
\begin{theoreme}
Soient $a < b$ des réels et $f \colon [a, b] \to \R$ une fonction continue. Alors $f$ est bornée sur $[a, b]$ et elle atteint ses bornes.
\begin{align*}
\exists n \in \R, \exists M \in \R \text{ tels que } \forall x \in [a, b], m \leq f(x) \leq M \text{ et } \exists x_0 \in [a, b], f(x_0) = m, \exists x_1 \in [a, b], f(x_1) = M
\end{align*}
\end{theoreme}
\end{graybox}
\par \noindent La preuve repose sur le théorème de Bolzano-Weierstrass.
\begin{remarque}
Il est important de considérer un intervalle fermé.
\end{remarque}

\begin{exemple}
La fonction $x \mapsto \frac{1}{x}$ est continue sur $]0, 1[$, mais pas bornée.
\end{exemple}

\par \noindent \textbf{Prolongement par continuité}
$I \subset \R$ un intervalle, $x_0 \in I$, $f \colon I \backslash \{x_0\} \to \R$.
\par \noindent On suppose que $\lim_{x \to x_0} f(x) = \ell$ existe. Alors la fonction
\begin{align*}
\tilde{f} \colon
\begin{cases}
I &\to \R \\
x &\mapsto
\begin{cases}
f(x) \text{ si } x \neq x_0 \\
\ell \text{ si } x = x_0
\end{cases}
\end{cases}
\end{align*}
On dit que $f$ est le prolongement par continuité de $f$ en $x_0$.

\begin{exemple}
Soit 
$
f \colon 
\begin{cases}
\R^* &\to \R \\
x &\mapsto \frac{\sin(x)}{x}
\end{cases}
$
continue sur $\R^*$. Peut-on la prolonger par continuité en $0$ ?
\\
Pour $x \neq 0$
\begin{align*}
\frac{\sin (x)}{x} = \frac{\sin (x) - \sin (0)}{x - 0}
\end{align*}
c'est le taux d'accroissement donc $\lim_{x \to 0} \frac{\sin (x)}{x} = \sin' (0) = \cos (0) = 1$.
On peut donc prolonger $f$ par continuité en posant
\begin{align*}
\tilde{f}(x) =
\begin{cases}
f(x) \text{ si } x \in \R^* \\
1 \text{ si } x = 0
\end{cases} 
\end{align*}
La fonction $\tilde{f}$ est continue sur $\R$.
\end{exemple}
