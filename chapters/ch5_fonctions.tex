\chapter{Continuité et limites de fonctions}

Soit $I$ un intervalle de $\R$ et $f \colon I \to \R$ une fonction.
Soit $x_0 \in I$ ou $x_0$ une borne de $I$ (possiblement $\pm \infty$).
On veut définir ce que veut dire 
\begin{align*}
    \lim_{x \to x_0} f(x) = \ell,\ \ell \in \R \text{ ou } \ell = \pm \infty
\end{align*}

\begin{graybox}
\begin{definition}[Limite d'une fonction en un point]~ 
\\
\textbf{1er cas} \\
$x_0 \in \R$, $\ell \in \R$ \\
On dit que $\displaystyle \lim_{x \to x_0} f(x) = \ell$
\begin{align*}
    \text{Si } \forall \varepsilon > 0, \exists \delta > 0 \text{ tel que si } |x -x_0| \leq \delta, \text{ alors } |f(x) - \ell| \leq \varepsilon
\end{align*}
\textbf{2e cas} \\
$x_0 \in \R$, $\ell = \pm \infty$ \\
On dit que $\displaystyle \lim_{x \to x_0} f(x) = +\infty$ 
\begin{align*}
    \text{Si } \forall A \in \R, \exists \delta > 0 \text{ tel que si } |x - x_0| \leq \delta \text{ alors } f(x) \geq A
\end{align*}
On dit que $\displaystyle \lim_{x \to x_0} f(x) = -\infty$ 
\begin{align*}
    \text{Si } \forall A \in \R, \exists \delta > 0 \text{ tel que si } |x - x_0| \leq \delta \text{ alors } f(x) \leq A
\end{align*}
\end{definition}
\end{graybox}

\begin{graybox}
    \begin{definition}[Limite d'une fonction en l'infini]~
        \\
\textbf{1er cas} \\
$x_0 = \pm \infty,\ \ell \in \R$ \\
On dit que $\displaystyle \lim_{x \to +\infty} f(x) = \ell $
\begin{align*}
    \text{Si } \forall \varepsilon > 0, \exists B \in \R \text{ tel que si } x \geq B \text{ alors } |f(x) - \ell| < \varepsilon
\end{align*}
On dit que $\displaystyle \lim_{x \to -\infty} f(x) = \ell $
\begin{align*}
    \text{ Si} \forall \varepsilon > 0, \exists B \in \R \text{ tel que si } x \leq B \text{ alors } |f(x) - \ell| \leq \varepsilon
\end{align*}
\textbf{2e cas} \\
$x_0 = \pm \infty$, $\ell = \pm \infty$ \\
On dit que $\displaystyle \lim_{x \to +\infty} = +\infty$ 
\begin{align*}
    \text{Si } \forall A \in \R, \exists B \in \R \text{ tel que si } x \geq B \text{ alors } f(x) \geq A
\end{align*}
On dit que $\lim_{x \to +\infty} = -\infty$
\begin{align*}
    \text{Si } \forall A \in \R, \exists B \in \R \text{ tel que si } x \geq B \text{ alors } f(x) \leq A
\end{align*}
On dit que $\displaystyle \lim_{x \to -\infty} f(x) = +\infty$ 
\begin{align*}
    \text{Si } \forall A \in \R, \exists B \in \R \text{ tel que si } x \leq B \text{ alors } f(x) \geq A
\end{align*}
on dit que $\displaystyle \lim_{x \to -\infty} f(x) = -\infty$
\begin{align*}
    \text{Si } \forall A \in \R, \exists B \in \R \text{ tel que si } x \leq B \text{ alors } f(x) \leq A
\end{align*}

\end{definition}
\end{graybox}
\begin{remarque}
    Lorsque $\displaystyle \lim_{x \to x_0} f(x) = 
        \begin{cases}
            +\infty \\
            \text{ou} \\
            -\infty
        \end{cases}$
    la droite verticale d'équation $x = x_0$ est une asymptote verticale au graphe de $f$
\end{remarque}

\begin{remarque}
    Lorsque $\displaystyle \lim_{x \to x_0} f(x) = \ell \text{ ou } \lim_{x \to x_0} f(x) = \ell$, la droite horizontale d'équation $y = \ell$ une asymptote horizontale au graphe de $f$
\end{remarque}

\begin{graybox}
    \begin{theoreme}[Caractérisation séquentielle des limites]
        Soit $I \subset \R$ un intervalle, $f \colon I \to \R$, $x_0 \in I$ ou une borne de $I$ (éventuellement $\pm \infty$), $\ell \in \R$ ou $\ell = +\infty$ ou $\ell = -\infty$.     
    \begin{align*}
        \text{Alors } \lim_{x \to x_0} f(x) = \ell \iff \text{Pour toute suite } (u_n)_{n \in \N} \text{ telle que } \lim_{n \to \infty} u_n = x_0, \text{ on a } \\ \lim_{n \to \infty} f(u_n) = \ell
    \end{align*}
    \end{theoreme}
\end{graybox}
